\documentclass[12pt,a4paper]{article}
\usepackage[polish]{babel}
\usepackage[T1]{fontenc}
\usepackage[utf8x]{inputenc}
% \usepackage{hyperref}
\usepackage{url}
\usepackage{graphicx}

\addtolength{\hoffset}{-1.5cm}
\addtolength{\marginparwidth}{-1.5cm}
\addtolength{\textwidth}{3cm}
\addtolength{\voffset}{-1cm}
\addtolength{\textheight}{2.5cm}
\setlength{\topmargin}{0cm}
\setlength{\headheight}{0cm}

\begin{document}

\title{Dokumentacja projektu Bazy danych}
\author{Nikodem Bulanda}
\date{\today}

\maketitle

\begin{center}
\textbf{LabMaster - System zarządzania laboratoriami diagnostycznymi} 
\end{center}

\begin{center}
Autorzy: Jakub Gwiżdż, Justyna Jachowicz, Kamil Oleś \\
\end{center}

\begin{center}
Prowadzący: mgr inż. Nikodem Bulanda
\end{center}

\newpage

\tableofcontents
\listoftables
\listoffigures

\newpage

\section{Tytył}
\textbf{System zarządzania laboratoriami diagnostycznymi}\\

\section{Nazwa robocza}
LabMaster System - \\

\newpage
\section{Cel projektu}
% Jednoznacznie określony stan w przyszłości, nie mylić z produktem końcowym\\

\hspace{0.60cm}Celem projektu "LabMaster System" jest stworzenie aplikacji, która umożliwi efektywne zarządzanie procesami związanymi z diagnostyką medyczną w laboratoriach. Projekt ma na celu dostarczenie produktu w postaci systemu informatycznego funkcjonującego w oparciu o kilka niezbędnych elementów. Zadaniem systemu jest efektywne zarządzenie wszystkimi aspektami związanymi z laboratoriami diagnostycznymi.

Zaczynając od rejestracji pacjentów i zleceniu badań, analizie wyników, aż po generowaniu raportów i faktur.Cel jest osiągany poprzez optymalizację procesów, minimalizację błędów oraz automatyzację tam, gdzie to możliwe. Istotnym aspektem jest również zapewnienie wysokich standardów bezpieczeństwa danych medycznych, zgodnie z regulacjami HIPAA/GDPR. System musi gwarantować poufność, integralność i dostępność danych pacjentów oraz badań laboratoryjnych.

Musi również cechować się doskonałą obsługą klienta, poprzez łatwy dostęp do usług diagnostycznych oraz szybkiego i precyzyjnego przetwarzania wyników badań. System powinien umożliwiać szybką rejestrację pacjentów, sprawdzenie zleceń NFZ, jak również generowanie raportów dostępnych dla pacjentów, lekarzy i innych zainteresowanych stron. 

Skuteczna kontrola - Projekt zakłada wprowadzenie systematycznej kontroli jakości procesów laboratoryjnych oraz zapewnienie reakcji na potencjalne odchylenia. Automatyczne informowanie użytkowników o ewentualnych błędach pozwoli na szybką korektę i minimalizację ryzyka.

Dlatego, system ma być elastyczny i łatwo integrowalny z istniejącymi systemami medycznymi, zapewniając możliwości dostępu poprzez urządzenia mobilne oraz integracje z zewnętrznymi systemami pozwoli to  na płynną wymianę danych oraz elastyczność operacyjną. 

Każdy projekt powinien mieć utworzony interfejs użytkownika, który będzie łatwy w obsłudze dla wszystkich aktorów zaangażowanych w proces diagnostyczny - od personelu laboratorium, przez personel administracyjny, po lekarzy i pacjentów.  usługi medyczne powinny być  szybkie i  precyzyjne, aby tego dokonać należy zwiększyć efektywność operacyjną Poprzez automatyzację procesów, monitorowanie wydajności oraz integrację zewnętrznymi systemami medycznymi.



\newpage
\section{Zakres}

\subsection{Analiza wymagań}
 (oraz ,,deasemblacja'' procesu osiągnięcia celu)
\\
 https://www.atd-software.pl/oferta/
\\ 

\begin{itemize}

\item Rejestracja zleceń i pacjentów:
      \begin{itemize}
      \item System ma umożliwiać łatwą rejestrację pacjentów.
      \item Implementacja tworzenia zleceń badań z możliwością dodawania szczegółów, sprawdzanie zleceń NFZ, obliczanie kosztów.
      \end{itemize}
      
\item Walidacja:
    \begin{itemize}
    \item System powinien zawierać mechanizmy walidacji danych wejściowych według istniejących wzorów systemowych, w celu uniknięcia błędów i zapewnienia dokładności informacji.
    \end{itemize}
    
\item Księgowość:
    \begin{itemize}
    \item System ma obejmować zarządzanie płatnościami, fakturowaniem oraz monitorowanie kosztów związanych z badaniami.
    \item System musi umożliwiać generowanie różnorodnych raportów dla pacjentów, lekarzy i innych zainteresowanych stron.
    \item Raport powinny być łatwo współdzielone, jak wysyłka mailem, druk, poczta itp.
    \end{itemize}

\item Pracownie analityczne/laboratorium:
    \begin{itemize}
    \item Skupienie na efektywnym zarządzaniu pracowniami analitycznymi, w tym planowaniu zadań i monitorowaniu postępu prac, dostęp do informacji o próbkach i zleconych badaniach
    \item Wprowadzenie modułu umożliwiającego zarządzanie danymi związanymi z badaniami mikrobiologicznymi, takimi jak wzory wypełnienia, automatyczna wstępna analiza wyników.
    \end{itemize}

\item  Kontrola jakości:
    \begin{itemize}
    \item Zapewnienie systematycznej kontroli jakości, zarządzania dokumentacją związaną z kontrolą jakości oraz reakcji na potencjalne odchylenia.
    \item Automatycznie informuje o możliwym błędzie zainteresowanego użytkownika.
    \end{itemize}

\item Magazyn:
    \begin{itemize}
    \item Implementacja skutecznego zarządzania inwentarzem i magazynem, z uwzględnieniem dostępności reagentów, sprzętu laboratoryjnego itp.
    \item Dokładny spis istniejących próbek przeznaczonych do badań.
    \end{itemize}


\end{itemize}



- // Rejestracja pacjenta/klienta [recepcjonistka] \\
 - Sprawdzenie zlecenia NFZ \\
 - lub sprzedaż badania \\
\\
- // Pobranie informacji o pacjencie (on podaje przy recepcji)\\
 - // Pobranie materiału (krew, mocz itp.) do badania [niektóre z próbnej pobrane w ośrodku]\\
 - Przyznali wewnętrznego identyfikatora próbce do badania, umieszczenie danych o badaniach do próbki do systemu\\
 -  Próba do magazynu, dane odnotowane do księgowości \\
 \\
- // Dane próbki i badani do wykonania są dostępne w grafiku pracownika  badań [laborant] \\
- Wykonacie przypisanych badań do próbki, wprowadzenie wyników do systemu [laborant] \\
\\
- // Dane przekazane do administracji [księgowa]\\
- Sprawdzenie prawidłowości danych w systemie (wykonanie wszystkich badań przypisanych próbką, możliwe błędy, zapóźnienia, ) \\
- Odnotowanie niezbędnych zmian w systemie, usunięcie próbek \\
- Wysyłka wyników do zainteresowanych stron (lekarz, pacjent itp.).\\
- Archiwizacja danych wykonanego zlecenia \\
\\
 Aktorzy:\\
 - recepcja\\
 - laboratorium\\
 - księgowość\\
 - admin\\
\\
 Co powinni wiedzieć/ móc zrobić 'aktorzy' by wykonać pracę:\\
 A.) Recepcja\\
 - Rejestracja danych nowych pacjentów\\
 - Przypisywanie badania do pacjentów, i związanych próbek\\
 - Zarządza dostępem do danych pacjentów.\\
 - Przegląda historię wykonanych już badań pacjenta.\\
 \\
 B.) Laboratorium \\
 - Plan badań (harmonogram) do wykonania przez wyznaczony personel\\
 - Dane o składanych próbkach.\\
 - Składanie precyzyjnych wyników badań do systemu.\\
 - Stan systemów i maszyn diagnostycznych niezbędnych do prowadzenia badań\\
 - Sprawdzenie rozkładu użyć maszyn/ terminów wykonania badań\\
 \\
 C.)Księgowość \\
 - Wprowadza jednolitych danych badań laboratoryjnych i pacjenta do system.\\
 - Generuje raporty dotyczące wyników z danych dostępnych wewnątrz systemu.\\
 - Odpieranie zakładanych zamówienia na badania laboratoryjne dla konkretnych pacjentów (NFZ).\\
 - Zarządza danymi dotyczącymi personelu medycznego.\\
 - Monitoruje stan zasobów laboratorium.\\


 





zadania aktorów cd.

recepcjoniskta:
Rejestruje nowych pacjentów.
Przypisuje badania do pacjentów.
Wprowadza wyniki badań laboratoryjnych.
Generuje raporty dotyczące wyników.
Zarządza dostępem do danych pacjentów.

Lekarz:
Przegląda wyniki badań pacjentów.
Składa zamówienia na badania laboratoryjne.
Dostęp do historii badań swoich pacjentów.

Pacjent:
Sprawdza wyniki swoich badań.
Rejestruje się na badania laboratoryjne.
Przegląda historię badań.

Personel Administracyjny:
Zarządza danymi dotyczącymi personelu medycznego.
Monitoruje stan zasobów laboratorium.

///////////////////////////

z chatu
Zarządzanie pacjentami:
Umożliw użytkownikom dodawanie, przeglądanie i aktualizowanie informacji o pacjentach.
Stwórz mechanizmy przypisywania badań do konkretnych pacjentów.

Zarządzanie badaniami:
Umożliw dodawanie nowych rodzajów badań laboratoryjnych.
Zapewnij funkcje związane z przypisywaniem badań do pacjentów.

Wprowadzanie i przetwarzanie wyników:
Dodaj funkcje wprowadzania wyników badań laboratoryjnych.
Zaimplementuj mechanizmy przetwarzania wyników i ich przechowywania w systemie.


\subsection{Wymagania funkcjonalne i niefunkcjonalne}
Funkcjonalne:
\begin{itemize}
\item Rejestracja Zleceń i Pacjentów:
  \begin{itemize}
  \item Aktorzy: recepcja
  \item Umożliwia rejestrację nowych pacjentów.
  \item Przypisuje badania do pacjentów i związanych próbek.
  \item Zarządza dostępem do danych pacjentów.
  \item Pozwala na przegląd historii badań pacjenta.
  \end{itemize}
  
\item Laboratorium:
  \begin{itemize}
  \item Aktorzy: laboratorium
  \item Udostępnia plan badań do wykonania przez wyznaczony personel.
  \item Zarządza danymi o składanych próbkach.
  \item Składa precyzyjne wyniki badań do systemu.
  \item Monitoruje stan systemów i maszyn diagnostycznych.
  \end{itemize}
  
\item Księgowość:
  \begin{itemize}
  \item Aktorzy: księgowość
  \item Wprowadza jednolite dane badań laboratoryjnych i pacjenta do systemu.
  \item Generuje raporty z danych wewnątrz systemu.
  \item Odpowiada za zakładane zamówienia na badania (NFZ).
  \item Zarządza danymi dotyczącymi personelu medycznego.
  \item Monitoruje stan zasobów laboratorium.
  \end{itemize}
\end{itemize}


Niefunkcjonalne:
\begin{itemize}

\item[--] Interfejs Użytkownika: Zapewnienie intuicyjnego interfejsu użytkownika dla wszystkich aktorów.
\item[--] Bezpieczeństwo Danych: Zastosowanie wysokich standardów bezpieczeństwa danych medycznych, zgodnie z normami HIPAA/GDPR.
\item[--] Automatyzacja: Automatyzacja procesów tam, gdzie to możliwe, w celu zwiększenia efektywności operacyjnej.
\item[--] Mobilność: Dostarczenie dostępu do systemu poprzez urządzenia mobilne dla elastyczności operacyjnej.
\item[--] Szkolenie Personelu: Dostarczenie funkcji pomocy i szkoleń online dla użytkowników systemu.
\item[--] Monitorowanie Wydajności: Implementacja narzędzi do monitorowania czasu przetwarzania próbek oraz efektywności operacyjnej laboratorium.
\item[--] Integracja z Zewnętrznymi Systemami: Zintegrowanie systemu z innymi systemami medycznymi dla płynnej wymiany danych.

\end{itemize}

\subsection{Diagram przypadków użycia i diagram przepływu (opcjonalny)}

\subsection{Dobór technologii}
- Język pisania: Java \\
- Bazy danych: MySQL \\
- Program: Intellij IDEA \\

\newpage
\section{Scenariusze}
(tytuł, numer, aktorzy, stan wejścia (warunki + dane), przebieg scenariusza, wynik, scenariusz alternatywny, jeśli istnieje)

\newpage
\section{Estymacja czasowa }
(poszczególnych zadań jak i określenie wymagań MVP oraz terminu końcowego oddania)
\\
- // Ustalenie zakresu teoretycznego: \\
    - technicznych wymagań program \\
    - wykorzystywanych technologi \\
    - ustaleń wstępnej budowy bazy danych \\
    - Wstępne ustalenie danych do obróbki \\
    - Stworzenie git-huba\\
- // Zatwierdzenie wybranych schematów danych i wprowadzenie oprawek:\\
    - Uzupełnienie brakujących elementów dokumentacji\\
    - Dodanie wykresów (wstępne)\\
    - Dodanie kilku prostych scenariuszy\\
- // Rozpoczęcie pracy nad programem\\
    - Podszkolenie pod względem pisma w języku programowania \\
    - Instalacja koniecznych programów i technologi\\
    - Zaczęcie projektowani interfejsu użytkownika\\
    - Stworzenie plików aplikacji i dodanie ich do git-huba \\
    - Stworzenie wstępnej bazy danych \\
- // Praca nag kodem aplikacji\\
    - Logowanie
    - 

\newpage
\section{Implementacja}

\newpage
\section{Testy i ich wyniki}
\section{Podsumowanie i bilans}
(MVP vs rzeczywistość)

\newpage
\section{Przykłady użycia elementów języka \LaTeX - nie wchodzi w zakres oddawanej dokumentacji stanowi jedynie przykład}
\begin{large}Powiększona czcionka.\end{large}
{\large To też jest powiększona czcionka.}

Jakiś nowy akapit.



To jest dobór technologii.
"Tekst w cudzysłowie podwójnym maszynowym" (wygląda nienaturalnie).
,,Tekst w cudzysłowie podwójnym''. ``Angielski cudzysłów''.

,,Twardą'' spację oznacza się znakiem tylda \~{} ($\sim$).
Mamy do dyspozycji trzy rodzaje myślników - ,,krótki'', -- ,,normalny'' i --- ,,długi''.

\label{nazwa_etykiety}

Przygotuj stronę w \textbf{HTML}'u, która jest ogłoszeniem o seminarium.
W lewym, górnym rogu strony umieść logo Wydziału Fizyki UW.
Podaj nazwę seminarium (np.~Seminarium Kosmologia i Fizyka Cząstek),
tytuł, imię i nazwisko  wygłaszającego seminarium,
instytucję której jest pracownikiem,
adres, numer sali, datę, godzinę.
 Dodaj także w punktach streszczenie wystąpienia\footnote{Jakaś informacja na marginesie.}.
 W zależności od stopnia ważności informacji, zróżnicuj rodzaj,
wielkość i typ czcionki -- Sekcja \ref{nazwa_etykiety}.

\section{Otoczenia}

\subsection{Formatowanie}

Żyjący w V wieku p.n.e.~prorok Malachiasz był autorem Księgi Malachiasza, będącej
ostatnią w grupie dwunastu ksiąg proroków mniejszych Starego Testamentu. Malachiasz jest
świętym Kościoła katolickiego i Cerkwi prawosławnej. Wśród badaczy nie ma zgody co do
tego, czy słowo Malachiasz to imię, tytuł proroka, czy też przypisana sobie przez
anonimowego autora rola posłańca Bożego. Podobne słowo w takim kontekście pojawia
się w Ml 2,7 i Ml 3,1. Rozbieżności mogły powstać za sprawą greckiego tłumacza, który
w Septuagincie przełożył z hebrajskiego Brzemię słowa Pańskiego w ręce Malachi na
(...) w ręce anioła tj. posła Jego, pozbawiając je jednocześnie cech
imienia własnego. Na tej podstawie Orygenes i Tertulian sądzili, że prorok był aniołem.

\begin{center}
Żyjący w V wieku p.n.e. prorok Malachiasz był autorem Księgi Malachiasza, będącej
ostatnią w grupie dwunastu ksiąg proroków mniejszych Starego Testamentu. Malachiasz jest
świętym Kościoła katolickiego i Cerkwi prawosławnej. Wśród badaczy nie ma zgody co do
tego, czy słowo Malachiasz to imię, tytuł proroka, czy też przypisana sobie przez
anonimowego autora rola posłańca Bożego. Podobne słowo w takim kontekście pojawia
się w Ml 2,7 i Ml 3,1. Rozbieżności mogły powstać za sprawą greckiego tłumacza, który
w Septuagincie przełożył z hebrajskiego Brzemię słowa Pańskiego w ręce Malachi na
(...) w ręce anioła tj. posła Jego, pozbawiając je jednocześnie cech
imienia własnego. Na tej podstawie Orygenes i Tertulian sądzili, że prorok był aniołem.
\end{center}

\begin{flushleft}
Żyjący w V wieku p.n.e. prorok Malachiasz był autorem Księgi Malachiasza, będącej
ostatnią w grupie dwunastu ksiąg proroków mniejszych Starego Testamentu. Malachiasz jest
świętym Kościoła katolickiego i Cerkwi prawosławnej. Wśród badaczy nie ma zgody co do
tego, czy słowo Malachiasz to imię, tytuł proroka, czy też przypisana sobie przez
anonimowego autora rola posłańca Bożego. Podobne słowo w takim kontekście pojawia
się w Ml 2,7 i Ml 3,1. Rozbieżności mogły powstać za sprawą greckiego tłumacza, który
w Septuagincie przełożył z hebrajskiego Brzemię słowa Pańskiego w ręce Malachi na
(...) w ręce anioła tj. posła Jego, pozbawiając je jednocześnie cech
imienia własnego. Na tej podstawie Orygenes i Tertulian sądzili, że prorok był aniołem.
\end{flushleft}

\begin{quote}
Żyjący w V wieku p.n.e.~prorok Malachiasz był autorem Księgi Malachiasza, będącej
ostatnią w grupie dwunastu ksiąg proroków mniejszych Starego Testamentu. Malachiasz jest
świętym Kościoła katolickiego i Cerkwi prawosławnej.

Wśród badaczy nie ma zgody co do
tego, czy słowo Malachiasz to imię, tytuł proroka, czy też przypisana sobie przez
anonimowego autora rola posłańca Bożego. Podobne słowo w takim kontekście pojawia
się w Ml 2,7 i Ml 3,1. Rozbieżności mogły powstać za sprawą greckiego tłumacza, który
w Septuagincie przełożył z hebrajskiego Brzemię słowa Pańskiego w ręce Malachi na
(...) w ręce anioła tj. posła Jego, pozbawiając je jednocześnie cech
imienia własnego. Na tej podstawie Orygenes i Tertulian sądzili, że prorok był aniołem.
\end{quote}

\begin{quotation}
Żyjący w V wieku p.n.e.~prorok Malachiasz był autorem Księgi Malachiasza, będącej
ostatnią w grupie dwunastu ksiąg proroków mniejszych Starego Testamentu. Malachiasz jest
świętym Kościoła katolickiego i Cerkwi prawosławnej.

Wśród badaczy nie ma zgody co do
tego, czy słowo Malachiasz to imię, tytuł proroka, czy też przypisana sobie przez
anonimowego autora rola posłańca Bożego. Podobne słowo w takim kontekście pojawia
się w Ml 2,7 i Ml 3,1. Rozbieżności mogły powstać za sprawą greckiego tłumacza, który
w Septuagincie przełożył z hebrajskiego Brzemię słowa Pańskiego w ręce Malachi na
(...) w ręce anioła tj. posła Jego, pozbawiając je jednocześnie cech
imienia własnego. Na tej podstawie Orygenes i Tertulian sądzili, że prorok był aniołem.
\end{quotation}

{\footnotesize
\begin{verbatim}
<html>
<head>
  <meta http-equiv="Content-type" content="text/html; charset=UTF-8" />
  <meta http-equiv="Content-language" content="pl" />
  <link href='style.css' rel='stylesheet' type='text/css' />
  <title>R. J. Wysocki</title>
</head>
\end{verbatim}
}

\subsection{Wypunktowanie i numeracja}

\begin{itemize}
\item[--] Żyjący w V wieku p.n.e.~prorok Malachiasz był autorem Księgi Malachiasza, będącej
  ostatnią w grupie dwunastu ksiąg proroków mniejszych Starego Testamentu. Malachiasz jest
  świętym Kościoła katolickiego i Cerkwi prawosławnej.
\item Drugi punkt.
  \begin{itemize}
  \item Pierwszy podpunkt.
  \item Drugi podpunkt.
  \end{itemize}
\item Trzeci punkt.
\item Czwarty punkt.
\end{itemize}

\begin{enumerate}
\item Żyjący w V wieku p.n.e.~prorok Malachiasz był autorem Księgi Malachiasza, będącej
  ostatnią w grupie dwunastu ksiąg proroków mniejszych Starego Testamentu. Malachiasz jest
  świętym Kościoła katolickiego i Cerkwi prawosławnej.
\item Drugi punkt.
  \begin{itemize}
  \item[--] Pierwszy podpunkt.
  \item[--] Drugi podpunkt.
  \end{itemize}
\item Trzeci punkt.
  \begin{enumerate}
  \item Pierwszy podpunkt.
  \item Drugi podpunkt..
    \begin{enumerate}
    \item Ala
    \item ma
    \item kota.
    \end{enumerate}
  \end{enumerate}
\item Czwarty punkt.
\end{enumerate}

\begin{description}
\item[nazwa 1] -- opis nazwy 1.
\item[nazwa 2] -- opis nazwy 2.
\item[nazwa 3] -- opis nazwy 3.  Opis może być dłuższy, niż jeden wiersz i warto
  zobaczyć co się wtedy stanie.
\end{description}

\subsection{Tabele}

\begin{table}[htb]
  \begin{tabular}{clr}
  {\bf Wyśrodkowanie} & {\bf Do lewej} & {\bf Do prawej} \\
  Treść & Treść & Treść \\
  Kolejny wiersz & Kolejnuy wiersz & Kolejny wiersz \\
  \end{tabular}
\caption{Tabela}
\label{tab:bez_ramek}
\end{table}

\begin{table}[htb]
  \begin{tabular}{c|l|r}
  {\bf Wyśrodkowanie} & {\bf Do lewej} & {\bf Do prawej} \\
  Treść & Treść & Treść \\
  Kolejny wiersz & Kolejnuy wiersz & Kolejny wiersz \\
  \end{tabular}
\caption{Tabela z liniami pionowymi między kolumnami}
\label{tab:pionowe}
\end{table}

\begin{table}[htb]
  \begin{tabular}{|c|l|r|}
  \hline
  {\bf Wyśrodkowanie} & {\bf Do lewej} & {\bf Do prawej} \\
  \hline
  \hline
  Treść & Treść & Treść \\
  \hline
  Kolejny wiersz & Kolejnuy wiersz & Kolejny wiersz \\
  \hline
  \end{tabular}
\caption{Tabela z liniami pionowymi między kolumnami i poziomymi między wierszami}
\label{tab:ramki}
\end{table}

\begin{table}[htb]
  \begin{tabular}{|c|l|p{6cm}|}
  \hline
  {\bf Wyśrodkowanie} & {\bf Do lewej} & {\bf Paragraf} \\
  \hline
  \hline
  Treść & Treść & Treść \\
  \hline
  Kolejny wiersz & Kolejnuy wiersz &
  Żyjący w V wieku p.n.e.~prorok Malachiasz był autorem Księgi Malachiasza, będącej
  ostatnią w grupie dwunastu ksiąg proroków mniejszych Starego Testamentu. \\
  \hline
  \end{tabular}
\caption{Tabela z dłuższym tekstem}
\label{tab:paragraf}
\end{table}

Na rysunku \ref{sinus} jest przedstawiony wykres funkcji sin(x).  W tablicach
\ref{tab:bez_ramek}, \ref{tab:pionowe}, \ref{tab:ramki}, \ref{tab:paragraf}
mamy przykłady zastosowania środowiska \verb#tabular#.

Odwołanie do literatury -- pierwsza pozycja w spisie \cite{Wikipedia}, moja strona
domowa \cite{RJW}.

\begin{figure}[htb!p]
% \includegraphics[width=0.9\textwidth]{sinus.pdf}
\caption{Wykres funkcji sin(x)}
\label{sinus}
\end{figure}

\begin{thebibliography}{9}
\bibitem{Wikipedia} {\it Pauli matrices}
  (\url{http://en.wikipedia.org/wiki/Pauli_matrices}).
\bibitem{RJW} {\it Moja strona} (\url{http://www.fuw.edu.pl/~rwys}).
\end{thebibliography}

\end{document}